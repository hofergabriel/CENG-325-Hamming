\documentclass[12pt]{article}
\usepackage{listings}
\usepackage{pdfpages}
\usepackage{amsmath}
%\usepackage[legalpaper, margin=1in]{geometry}
\begin{document}
\begin{titlepage}
   \begin{center}
       \vspace*{1cm}
       \Large
       Programming Assignment 1
       \normalsize

       \vspace{0.5cm}

       Author: Gabriel Hofer

       \vspace{0.5cm}

       CSC-325

       \vspace{0.5cm}

       Instructor: Dr. Karlsson

       \vspace{0.5cm}

       Due: September 25, 2020

       \vfill

       Computer Science and Engineering\\
       South Dakota School of Mines and Technology\\
   \end{center}
\end{titlepage}
%------------------------------------------------------------------------------------


\newpage
We randomly create vector $p$ containing either 4 or 11 data bits by calling makeMessage.
\[
  p = 
  \begin{pmatrix}
    d_1 \\
    d_2 \\
    d_3 \\
    d_4
  \end{pmatrix}
  =
  \begin{pmatrix}
    1 \\
    0 \\
    1 \\
    1
  \end{pmatrix}
\]
\begin{lstlisting}[frame=single,language=Python,caption=makeMessage \label{code:makeMessage}]
def makeMessage(n): 
  return np.random.randint(2, size=(n, 1))
\end{lstlisting}

Then we encode $p$ by pre-multiplying $p$ by $G$ modulo 2 in the encode function.
\[
  x = Gp = 
  \begin{pmatrix}
    1 & 1 & 0 & 1 \\  
    1 & 0 & 1 & 1 \\ 
    1 & 0 & 0 & 0 \\  
    0 & 1 & 1 & 1 \\ 
    0 & 1 & 0 & 0 \\  
    0 & 0 & 1 & 0 \\  
    0 & 0 & 0 & 1 
  \end{pmatrix}
  \begin{pmatrix}
    1 \\
    0 \\
    1 \\
    1
  \end{pmatrix} 
  =
  \begin{pmatrix}
    2 \\
    3 \\
    1 \\
    2 \\
    0 \\
    1 \\
    1
  \end{pmatrix} 
  = 
  \begin{pmatrix}
    0 \\
    1 \\
    1 \\
    0 \\
    0 \\
    1 \\
    1 \\
  \end{pmatrix} 
\]
\begin{lstlisting}[frame=single,language=Python,caption=encode \label{code:makeMessage}]
def encode(G,p):
  return np.matmul(G,p) & 1
\end{lstlisting}

We sometimes create an error (flip a bit) in $p$ using the makeError function.  
\begin{lstlisting}[frame=single,language=Python,caption=makeError \label{code:makeMessage}]
def makeError(p): 
  if random.randint(0,1): return p
  rdm=random.randint(0,p.shape[0]-1)
  p[rdm,0]=p[rdm,0]^1;
  return p
\end{lstlisting}

We check to see where errors occured by pre-multiplying $x$ by the parity-check matrix, $H$. 
This is done in parityCheck function.
\[
  z = Hr = 
  \begin{pmatrix}
    1 & 0 & 1 & 0 & 1 & 0 & 1 \\
    0 & 1 & 1 & 0 & 0 & 1 & 1 \\
    0 & 0 & 0 & 1 & 1 & 1 & 1 
  \end{pmatrix}
  \begin{pmatrix}
    0 \\
    1 \\
    1 \\
    0 \\
    0 \\
    1 \\
    1
  \end{pmatrix} 
  =
  \begin{pmatrix}
    2 \\
    4 \\
    2
  \end{pmatrix} 
  =
  \begin{pmatrix}
    0 \\
    0 \\
    0
  \end{pmatrix}
\]
\begin{lstlisting}[frame=single,language=Python,caption=parityCheck \label{code:makeMessage}]
def parityCheck(H,r): 
  return np.matmul(H,r) & 1
\end{lstlisting}

Correct the error by flipping the bit that was incorrect accoring to the syndrome vector $z$.
\begin{lstlisting}[frame=single,language=Python,caption=correctError \label{code:makeMessage}]
def correctError(z,r): 
  loc=0
  for i in range(0,z.shape[0]):
    loc+=z[i,0]*pow(2,i)
  if loc==0: return r
  r[loc-1,0]=r[loc-1,0]^1;
  return r
\end{lstlisting}

Finally, we decode the message by pre-multiply the encoded message by a decoding matrix $R$.
\[
  p_r = Rr = 
  \begin{pmatrix}
    0 & 0 & 1 & 0 & 0 & 0 & 0 \\
    0 & 0 & 0 & 0 & 1 & 0 & 0 \\
    0 & 0 & 0 & 0 & 0 & 1 & 0 \\
    0 & 0 & 0 & 0 & 0 & 0 & 1 
  \end{pmatrix} 
  \begin{pmatrix}
    0 \\
    1 \\
    1 \\
    0 \\
    0 \\
    1 \\
    1 
  \end{pmatrix} 
  =
  \begin{pmatrix}
    1 \\
    0 \\
    1 \\
    1
  \end{pmatrix} 
\]
\begin{lstlisting}[frame=single,language=Python,caption=decodeMessage \label{code:makeMessage}]
def decodeMessage(R,r): 
  return np.matmul(R,r)
\end{lstlisting}

Below is our main function. We hardcoded $G$, $H$, and $R$ according to 
which mode was entered by the user. Then these three matrices are passed to functions 
encode, parityCheck, and decode, respectively. I created the (15,11) matrices by 
looking at the bit patterns in the table in the writeup.
\begin{lstlisting}[frame=single,language=Python,caption=main \label{code:makeMessage}]
def main():
  # enter mode: either (7,4) or (15,11)
  mode=input("Enter mode: ")
  if mode=="H1511": 
    pLen=11
    G=np.array([
      [1,1,0,1,1,0,1,0,1,0,1], \
      [1,0,1,1,0,1,1,0,0,1,1], \
      [1,0,0,0,0,0,0,0,0,0,0], \
      [0,1,1,1,0,0,0,1,1,1,1], \
      [0,1,0,0,0,0,0,0,0,0,0], \
      [0,0,1,0,0,0,0,0,0,0,0], \
      [0,0,0,1,0,0,0,0,0,0,0], \
      [0,0,0,0,1,1,1,1,1,1,1], \
      [0,0,0,0,1,0,0,0,0,0,0], \
      [0,0,0,0,0,1,0,0,0,0,0], \
      [0,0,0,0,0,0,1,0,0,0,0], \
      [0,0,0,0,0,0,0,1,0,0,0], \
      [0,0,0,0,0,0,0,0,1,0,0], \
      [0,0,0,0,0,0,0,0,0,1,0], \
      [0,0,0,0,0,0,0,0,0,0,1]])
    H = np.array([ \
      [1,0,1,0,1,0,1,0,1,0,1,0,1,0,1], \
      [0,1,1,0,0,1,1,0,0,1,1,0,0,1,1], \
      [0,0,0,1,1,1,1,0,0,0,0,1,1,1,1], \
      [0,0,0,0,0,0,0,1,1,1,1,1,1,1,1], \
      ])
    R = np.array([ \
     [0,0,1,0,0,0,0,0,0,0,0,0,0,0,0], \
     [0,0,0,0,1,0,0,0,0,0,0,0,0,0,0], \
     [0,0,0,0,0,1,0,0,0,0,0,0,0,0,0], \
     [0,0,0,0,0,0,1,0,0,0,0,0,0,0,0], \
     [0,0,0,0,0,0,0,0,1,0,0,0,0,0,0], \
     [0,0,0,0,0,0,0,0,0,1,0,0,0,0,0], \
     [0,0,0,0,0,0,0,0,0,0,1,0,0,0,0], \
     [0,0,0,0,0,0,0,0,0,0,0,1,0,0,0], \
     [0,0,0,0,0,0,0,0,0,0,0,0,1,0,0], \
     [0,0,0,0,0,0,0,0,0,0,0,0,0,1,0], \
     [0,0,0,0,0,0,0,0,0,0,0,0,0,0,1]])
  else : 
    pLen=4
    G=np.array([ \
      [1,1,0,1], \
      [1,0,1,1], \
      [1,0,0,0], \
      [0,1,1,1], \
      [0,1,0,0], \
      [0,0,1,0], \
      [0,0,0,1]])
    H = np.array([ \
      [1,0,1,0,1,0,1], \
      [0,1,1,0,0,1,1], \
      [0,0,0,1,1,1,1]])
    R = np.array([ \
      [0,0,1,0,0,0,0], \
      [0,0,0,0,1,0,0], \
      [0,0,0,0,0,1,0], \
      [0,0,0,0,0,0,1]])

  # generate random message vector, p, of length 4 or 11 
  p=makeMessage(pLen);
  print("Message          : "+str(p.transpose()[0]))

  # encode (make send vector)
  x=encode(G,p)
  print("Send Vector      : "+str(x.transpose()[0]))

  # modify the vector to simulate an error or not
  r=makeError(x)
  print("Recieved Message : "+str(r.transpose()[0]))

  # Parity Check
  z=parityCheck(H,r)
  print("Parity Check     : "+str(z.transpose()[0]));

  # Error Correction
  corrected=correctError(z,r)
  print("Corrected Message: "+str(corrected.transpose()[0]))

  # Decode Message 
  pr=decodeMessage(R,x)
  print("Decoded Message  : "+str(pr.transpose()[0]));
\end{lstlisting}

\subsection*{Usage:}
Hamming.py can be run in Ubuntu using the python3 command in the terminal. 

\begin{lstlisting}[frame=single,language=Python,caption=main \label{code:makeMessage}]
  $ python3 Hamming.py
\end{lstlisting}

\end{document}

